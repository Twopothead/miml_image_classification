\chapter{总结与展望} 
% \pagenumbering{arabic} % 阿拉伯数字页码

\section{致谢}
感谢姜老师的耐心指导和方向上的提醒,我的问题是理论知识不扎实就动手干,
这样的后果常常是陷入迷茫,没有方向,常常纠结下一步要写什么代码,这是理论知识欠缺的表现.

最初打算写本项目是因为组队参加"龙芯杯"遭遇的彻底失败,这次失败使我意识到自己与他人的巨大差距,发现自己
学的课程没有联系起来,从而形成解决实际问题的综合能力,觉得需要做一个接近底层的综合性项目.感谢赵老师的
悉心指点与支持鼓励.

还要感谢朋友以及同学们项目中带给的大力支持和帮忙,给我带来极大的启发。
也要感谢参考文献中的作者们,通过他们的文章或著作,使我对项目有了很好的出发点。

\section{总结}

个人觉得本次课程设计难度比较大,对我是一次综合性考验.课设所用知识包括
操作系统原理、微机原理、汇编语言、数据结构、Linux命令等,还要涉及GAS汇编、Makefile编写、GRUB格式、
ATA硬盘串口驱动等零碎知识,这些做项目之前也都不懂,需要查阅很多资料.当程序出现奇怪的bug调不出来时,
一个人常常感到莫大的孤独.

文件管理虽然是相对独立的一章,但若要想在物理的机器上做出来,必须先写设备驱动,要键盘输入有反应,屏幕能够响应,
还得写中断处理程序和字符设备驱动,在此之前要先正确设置好gdt表和idt表,内存管理也需要能够malloc和free,
本系统还完成了简单的连续内存分配和对8259A,8253或其兼容芯片的编程.
所有这些必须正确做完才能进入文件系统的编写.怎样才能符合grub规范,
如何才能读写物理IDE硬盘扇区,如何写串口驱动.这些在我做此次课设之前,我也不清楚.
我阅读了文献和源代码之后,才有初步的认识,自己在编写代码和调试bug的过程中,
才逐步深入了解.当我遇到困难做不下去时,就多找几个这方面的材料对比着看.

当有人问Linus如何深入理解minix内核时,他提出"RTFSC":去读源码吧.这里RTFSC体现了阅读源码的艰辛和重要性.
.
在操作系统的学习中,我一直结合赵炯 那本书阅读Linux0.11源码,虽然艰难,但有时也会感到很有趣,
有时会发现比较实现起来很复杂的一件事,Linux代码几笔带过,实现得简洁有力,这是多么精妙!
Linux0.11是适合初学者的源码,同样的还有用于教育目的的经典的xv6内核,
国内有在jos和xv6基础上改的ucore实验,也可以看看.
我在课程设计进行中部分阅读了其他不知名的小kernel.真正仔细研究的是Linux0.11,xv6和ucore这三套源码,
阅读源码不会一帆风顺,当我遇到困难时,就把三个对比着看,看他们面对同样问题是如何处理的,从中收获很多.
Github和osdev网站也给我帮助.

网上关于简易内核编写的资料并不是很少,但关于文件系统的实现上,大多是实现已有文件系统
,最常见的是基于fat32,这其实是比较简单的,还有的是完全实现ext2甚至ext4文件系统,这个难度相当大,
我认为这不是普通学生能够在短时间做出来的.
实现一个自定义的文件系统难度应该在二者之间,也还是可以做到的,但在实践中越往后写下去越难找到可参考的资料,
水平有限,我也花费了很多时间.

在这一点上Linux0.11实现的是minix文件系统,xv6和ucore实现的是sfs文件系统,
由于minix文件系统和sfs都是已知的文件系统,这几个在交叉开发时可以采用mkfs命令做出一个磁盘镜像
,然后可以考虑和内核镜像做成系统集成盘.但这情况不适用于我们的RiOS,因为我的文件系统是自定义的,
除非自己写个类似mkfs的程序,否则想用mkfs命令做出相应磁盘镜像是不可能的.
最后,我采用系统探测磁盘并格式化的方法解决了这一问题.
至于空闲块成组链接,我也是反复研究课本,弄懂理论后摸索着写代码.

\subsection{存在问题}

"得失寸心知",为此项目颇费了一番精力,其中我也走过很多弯路,快乐与痛苦、优点与不足自己是最清楚的.
这里谈谈缺陷.本项目基本完整地实现了操作系统中文件系统部分,但是严格地说它并不能算得上一个整体结构完整的
简单内核.因为个人水平和时间精力有限,并没有实现进程管理,这是遗憾.内存管理使用了较为原始的方法,虽然
能够支持系统的运行,但不利于多进程的管理.

\subsection{展望}

不知道是否有同学对继续编写此项目有兴趣,如果真的有,那我将很高兴.
目前内核要做的工作很明显,一是把内存管理由最初的原始方法替换为段页式内存管理,
有了多进程在内存管理方面的基础,可以考虑实现真正的进程调度.后者如何实现,前面已经有相关论述,
应该说后者的基础工作(除了段页式内存管理),现在还是有的.若完成了这些,整个项目应该能够大体覆盖书上的
一些知识点了.还在网上看过\href{https://legacy.gitbook.com/book/nju-ics/ics2017-programming-assignment/details}{ics-pa实验},比较有趣
,我也挺想移植一个小游戏到我们的内核上去,不过限于水平和时间精力,没有付诸行动.
未来合适的时候,我可能还会在\href{https://github.com/Twopothead/baby-rios}{github}继续发展此项目,如果有想法,欢迎与我交流.






% \clearpage